% Options for packages loaded elsewhere
% Options for packages loaded elsewhere
\PassOptionsToPackage{unicode}{hyperref}
\PassOptionsToPackage{hyphens}{url}
\PassOptionsToPackage{dvipsnames,svgnames,x11names}{xcolor}
%
\documentclass[
  letterpaper,
  DIV=11,
  numbers=noendperiod]{scrreprt}
\usepackage{xcolor}
\usepackage{amsmath,amssymb}
\setcounter{secnumdepth}{5}
\usepackage{iftex}
\ifPDFTeX
  \usepackage[T1]{fontenc}
  \usepackage[utf8]{inputenc}
  \usepackage{textcomp} % provide euro and other symbols
\else % if luatex or xetex
  \usepackage{unicode-math} % this also loads fontspec
  \defaultfontfeatures{Scale=MatchLowercase}
  \defaultfontfeatures[\rmfamily]{Ligatures=TeX,Scale=1}
\fi
\usepackage{lmodern}
\ifPDFTeX\else
  % xetex/luatex font selection
\fi
% Use upquote if available, for straight quotes in verbatim environments
\IfFileExists{upquote.sty}{\usepackage{upquote}}{}
\IfFileExists{microtype.sty}{% use microtype if available
  \usepackage[]{microtype}
  \UseMicrotypeSet[protrusion]{basicmath} % disable protrusion for tt fonts
}{}
\makeatletter
\@ifundefined{KOMAClassName}{% if non-KOMA class
  \IfFileExists{parskip.sty}{%
    \usepackage{parskip}
  }{% else
    \setlength{\parindent}{0pt}
    \setlength{\parskip}{6pt plus 2pt minus 1pt}}
}{% if KOMA class
  \KOMAoptions{parskip=half}}
\makeatother
% Make \paragraph and \subparagraph free-standing
\makeatletter
\ifx\paragraph\undefined\else
  \let\oldparagraph\paragraph
  \renewcommand{\paragraph}{
    \@ifstar
      \xxxParagraphStar
      \xxxParagraphNoStar
  }
  \newcommand{\xxxParagraphStar}[1]{\oldparagraph*{#1}\mbox{}}
  \newcommand{\xxxParagraphNoStar}[1]{\oldparagraph{#1}\mbox{}}
\fi
\ifx\subparagraph\undefined\else
  \let\oldsubparagraph\subparagraph
  \renewcommand{\subparagraph}{
    \@ifstar
      \xxxSubParagraphStar
      \xxxSubParagraphNoStar
  }
  \newcommand{\xxxSubParagraphStar}[1]{\oldsubparagraph*{#1}\mbox{}}
  \newcommand{\xxxSubParagraphNoStar}[1]{\oldsubparagraph{#1}\mbox{}}
\fi
\makeatother


\usepackage{longtable,booktabs,array}
\usepackage{calc} % for calculating minipage widths
% Correct order of tables after \paragraph or \subparagraph
\usepackage{etoolbox}
\makeatletter
\patchcmd\longtable{\par}{\if@noskipsec\mbox{}\fi\par}{}{}
\makeatother
% Allow footnotes in longtable head/foot
\IfFileExists{footnotehyper.sty}{\usepackage{footnotehyper}}{\usepackage{footnote}}
\makesavenoteenv{longtable}
\usepackage{graphicx}
\makeatletter
\newsavebox\pandoc@box
\newcommand*\pandocbounded[1]{% scales image to fit in text height/width
  \sbox\pandoc@box{#1}%
  \Gscale@div\@tempa{\textheight}{\dimexpr\ht\pandoc@box+\dp\pandoc@box\relax}%
  \Gscale@div\@tempb{\linewidth}{\wd\pandoc@box}%
  \ifdim\@tempb\p@<\@tempa\p@\let\@tempa\@tempb\fi% select the smaller of both
  \ifdim\@tempa\p@<\p@\scalebox{\@tempa}{\usebox\pandoc@box}%
  \else\usebox{\pandoc@box}%
  \fi%
}
% Set default figure placement to htbp
\def\fps@figure{htbp}
\makeatother





\setlength{\emergencystretch}{3em} % prevent overfull lines

\providecommand{\tightlist}{%
  \setlength{\itemsep}{0pt}\setlength{\parskip}{0pt}}



 


\KOMAoption{captions}{tableheading}
\makeatletter
\@ifpackageloaded{bookmark}{}{\usepackage{bookmark}}
\makeatother
\makeatletter
\@ifpackageloaded{caption}{}{\usepackage{caption}}
\AtBeginDocument{%
\ifdefined\contentsname
  \renewcommand*\contentsname{Table of contents}
\else
  \newcommand\contentsname{Table of contents}
\fi
\ifdefined\listfigurename
  \renewcommand*\listfigurename{List of Figures}
\else
  \newcommand\listfigurename{List of Figures}
\fi
\ifdefined\listtablename
  \renewcommand*\listtablename{List of Tables}
\else
  \newcommand\listtablename{List of Tables}
\fi
\ifdefined\figurename
  \renewcommand*\figurename{Figure}
\else
  \newcommand\figurename{Figure}
\fi
\ifdefined\tablename
  \renewcommand*\tablename{Table}
\else
  \newcommand\tablename{Table}
\fi
}
\@ifpackageloaded{float}{}{\usepackage{float}}
\floatstyle{ruled}
\@ifundefined{c@chapter}{\newfloat{codelisting}{h}{lop}}{\newfloat{codelisting}{h}{lop}[chapter]}
\floatname{codelisting}{Listing}
\newcommand*\listoflistings{\listof{codelisting}{List of Listings}}
\makeatother
\makeatletter
\makeatother
\makeatletter
\@ifpackageloaded{caption}{}{\usepackage{caption}}
\@ifpackageloaded{subcaption}{}{\usepackage{subcaption}}
\makeatother
\usepackage{bookmark}
\IfFileExists{xurl.sty}{\usepackage{xurl}}{} % add URL line breaks if available
\urlstyle{same}
\hypersetup{
  pdftitle={My Personal Reviews},
  pdfauthor={Faizal Ali},
  colorlinks=true,
  linkcolor={blue},
  filecolor={Maroon},
  citecolor={Blue},
  urlcolor={Blue},
  pdfcreator={LaTeX via pandoc}}


\title{My Personal Reviews}
\usepackage{etoolbox}
\makeatletter
\providecommand{\subtitle}[1]{% add subtitle to \maketitle
  \apptocmd{\@title}{\par {\large #1 \par}}{}{}
}
\makeatother
\subtitle{Portfolio Asesmen II-2100 KIPP}
\author{18224007 Faizal Ali}
\date{2025-09-19}
\begin{document}
\maketitle

\renewcommand*\contentsname{Table of contents}
{
\hypersetup{linkcolor=}
\setcounter{tocdepth}{2}
\tableofcontents
}

\bookmarksetup{startatroot}

\chapter*{Hello}\label{hello}
\addcontentsline{toc}{chapter}{Hello}

\markboth{Hello}{Hello}

Website ini disusun sebagai pemenuhan kewajiban akademis untuk mata
kuliah II2100 Komunikasi Interpersonal dan Publik di Program Studi
Sistem dan Teknologi Informasi ITB. Melampaui tujuan tersebut,
portofolio ini saya maksudkan sebagai sebuah refleksi jujur atas proses
pembelajaran saya. Harapan saya, setiap bagian di dalamnya dapat
mengilustrasikan perkembangan saya, baik dalam hal pengetahuan, pola
pikir, maupun pemahaman diri.

\bookmarksetup{startatroot}

\chapter{UTS-1 All About Me}\label{uts-1-all-about-me}

Halo, perkenalkan nama saya \textbf{Faizal Ali}. Saya seorang mahasiswa
program studi \textbf{Sistem dan Teknologi Informasi (STI) di ITB}, yang
berasal dari \textbf{Kota Tegal}. Saya adalah pribadi yang memiliki
antusiasme tinggi untuk terus mempelajari hal-hal baru. Bagi saya,
komunikasi bukan hanya tentang cara menyampaikan apa yang ingin kita
ucapkan. Esensi sesungguhnya adalah bagaimana caranya agar pesan yang
kita sampaikan dapat dirasakan dengan jelas oleh lawan bicara. Ini
adalah tentang penyampaian makna secara utuh.

Sejalan dengan semangat saya mempelajari hal baru, saat ini saya sedang
tertarik untuk mendalami bidang \emph{data science}. Sepintas, bidang
ini mungkin tampak tidak berhubungan dengan ilmu komunikasi. Namun, saya
mulai menemukan sesuatu yang menarik. Saya mulai memahami bahwa
tantangan terbesar dalam data science bukanlah sekadar menemukan pola
dari angka. Tantangan sesungguhnya adalah bagaimana menyampaikan insight
data tersebut. Percuma kita memiliki model yang akurat jika kita tidak
bisa menyajikannya dengan cara yang dapat dipahami oleh orang lain yang
mungkin tidak memiliki latar belakang teknis.

Maka dari itu, bagi saya, proses belajar \emph{data science} dan proses
belajar komunikasi adalah satu kesatuan. Sembari saya belajar bagaimana
cara menyampaikan insight data secara efektif, saya juga terus belajar
hal yang paling penting dalam kehidupan yaitu, \emph{mencari cara
terbaik untuk bisa dipahami orang lain}.

\bookmarksetup{startatroot}

\chapter{UTS-2 My Songs for You}\label{uts-2-my-songs-for-you}

Sebagai manusia, kita hidup dalam kereta kita masing-masing. Ada yang
namanya kereta ekspektasi hingga kereta target. Seringkali, kita merasa
seperti sedang tidak mengemudikan kereta itu. Kita hanya penumpang yang
terbawa arus. Dan kereta itu rasanya kencang sekali. Saking kencangnya,
kita sering lupa untuk sekadar bernapas. Kita lupa melihat ke luar
jendela. Kita tidak lagi menikmati pemandangannya; kita hanya fokus
cemas pada stasiun tujuan berikutnya.

Bagi saya, lagu Nadin Amizah ini adalah sebuah jeda. Lagu ini
seakan-akan menjadi sahabat yang menepuk pundak kita dan berkata,
``Tidak apa-apa jika kamu merasa lelah. Tidak apa-apa jika kamu merasa
ini terlalu cepat.'' Lagu ini adalah pengingat bahwa kita bukanlah mesin
yang dirancang untuk produktivitas tanpa henti. Kita adalah manusia yang
butuh waktu untuk merasa, untuk hadir, dan untuk sekadar ada.

\url{https://www.youtube.com/watch?v=efQ5I4Ij0Gg}

\bookmarksetup{startatroot}

\chapter{UTS-3 My Stories for You}\label{uts-3-my-stories-for-you}

Halo, saya Faizal Ali, atau biasa dipanggil Faiz. Saat ini saya sedang
menempuh pendidikan sebagai mahasiswa di Sistem dan Teknologi Informasi
ITB. Tapi, jujur saja, ini bukanlah rencana A saya dari awal. Cerita
saya mungkin mirip dengan banyak orang lain: tentang bagaimana sebuah
rencana yang gagal justru menuntun kita ke jalan yang lebih baik,
meskipun awalnya tidak terasa begitu.

Semua bermula di hari pengumuman SNBT. Target saya saat itu jelas
sekali, Ilmu Komputer UI. Saya sudah membayangkan banyak hal yang akan
saya lakukan di sana. Saat saya buka pengumumannya, saya ditolak.
Rasanya berat sekali. Saya tidak akan bilang saya baik-baik saja, karena
sejujurnya saya \emph{sangat demotivated}. Apalagi saat melihat
teman-teman lain satu per satu mengunggah kabar gembira mereka. Tentu
saya ikut senang, tapi di dalam hati saya mulai \emph{takut}. Takut
kalau semua usaha saya selama ini sia-sia.

Mau tidak mau, saya tahu saya harus lanjut berjuang. Meratapi penolakan
itu tidak akan mengubah apa-apa. Saya kumpulkan lagi sisa-sisa semangat
saya, saya fokus untuk ujian mandiri. Saya mendaftar SM ITB untuk STEI-K
dan UTUL UGM untuk Statistika.

Lalu kabar baik pertama datang, saya diterima di STEI-K ITB. Rasanya
\emph{lega sekali}, seolah beban berat di pundak saya terangkat.
Ketakutan saya mulai hilang. Beberapa minggu setelahnya, kabar baik
kedua datang, saya juga diterima di Statistika UGM. Rasanya sedikit
sureal. Dari yang tadinya takut tidak dapat apa-apa, saya justru diberi
dua pilihan yang sama-sama hebat.

Di titik inilah saya merasa terbentuk. Saya tidak lagi memilih karena
gengsi atau ikut-ikutan. Saya benar-benar duduk dan memikirkannya. Saya
membandingkan kurikulum STEI-K, melihat jurusan yang ditawarkan, dan
saya merasa cocok. Saya merasa di sinilah saya bisa lebih bertumbuh.
Statistika UGM tentu pilihan yang sangat baik, tapi hati saya lebih
mantap ke STEI-K.

Sekarang, melihat ke belakang, saya jadi sadar. Penolakan dari Ilmu
Komputer UI itu bukan akhir dari segalanya. Itu hanya sebuah benturan
yang perlu saya alami. Saya terbentur dari apa yang saya kira saya
inginkan, agar saya bisa terbentuk di tempat yang ternyata saya
butuhkan.

\begin{quote}
Kalau ada teman-teman yang sedang di posisi yang sama, saya hanya ingin
bilang, penolakan itu rasanya memang sakit, tapi itu \emph{bukan vonis
gagal}. Saya percaya bahwa itu hanyalah cara Tuhan untuk mengarahkan
kita ke jalan yang lebih tepat.
\end{quote}

\bookmarksetup{startatroot}

\chapter{UTS-4 My SHAPE (Spiritual Gifts, Heart, Abilities, Personality,
Experiences)}\label{uts-4-my-shape-spiritual-gifts-heart-abilities-personality-experiences}

\section{Ringkasan}\label{ringkasan}

Peran Inti:

Analis Data (Data Analyst) / Saintis Data Junior yang berlatar belakang
pemikiran sistem (STI).

Misi:

Mengintegrasikan analisis data kuantitatif (Data Science) dengan
pemahaman proses dan sistem (STI) untuk mengekstrak wawasan yang relevan
dan merancang solusi berbasis data yang memecahkan masalah
bisnis/organisasi.

Kekuatan Utama:

Pemikiran analitis-kuantitatif, Data storytelling, Pemrograman data
(Python/SQL), Pemodelan dasar, Riset mandiri, Pemahaman alur sistem.

Dampak yang Dituju:

Keputusan yang didukung data (data-driven decisions), Wawasan (insights)
yang dapat ditindaklanjuti, Kontribusi pada tim sebagai penjembatan
antara data teknis dan kebutuhan sistem/manusia.

\subsection{Peta SHAPE (singkat):}\label{peta-shape-singkat}

\begin{itemize}
\tightlist
\item
  \textbf{S --- Panggilan Inti:} Analisis \& Sintesis, Keingintahuan
  (Curiosity), Integritas (Data), Pengenalan Pola (Pattern Recognition).
\item
  \textbf{H --- Minat \& Gairah:} Data Science, Machine Learning,
  Visualisasi Data, SQL, Storytelling with Data, dan penerapan analisis
  data untuk memahami proses bisnis dan sistem informasi (konteks STI).
\item
  \textbf{A --- Abilities (Kemampuan):} Analisis data kuantitatif,
  Pemrograman Data (Python: Pandas, Scikit-learn), Database (SQL),
  Visualisasi Data, Riset \& Pembelajaran Mandiri, Dasar-dasar Analisis
  Sistem \& Proses (dari STI).
\item
  \textbf{P --- Personality (Gaya Kerja):} Metodis \& terstruktur,
  analitis \& detail, mandiri \& proaktif (self-driven), reflektif,
  pragmatis.
\item
  \textbf{E --- Experiences (Pengalaman Pembentuk):} Proyek akademis di
  STI (Basis Data, Analisis Proses); Pengalaman ``benturan'': menemukan
  passion pada Data Science (lewat kursus online/lomba); Pengalaman
  ``terbentuk'': Proyek di mana berhasil menggabungkan data science
  dalam konteks masalah sistem informasi.
\end{itemize}

\section{1) S --- Panggilan Inti (Core
Calling)}\label{s-panggilan-inti-core-calling}

\begin{itemize}
\tightlist
\item
  \textbf{Analisis \& Sintesis:} Kemampuan alami untuk mengurai masalah
  besar menjadi komponen data yang logis dan menyatukannya kembali
  sebagai sebuah wawasan (insight).
\item
  \textbf{Keingintahuan (Curiosity):} Dorongan kuat untuk terus bertanya
  ``mengapa'' pada data (``Why does this pattern exist?''). Ini adalah
  bahan bakar utama untuk riset dan eksplorasi data.
\item
  \textbf{Integritas:} Kebutuhan untuk jujur pada data. Menyajikan
  temuan apa adanya, mengakui keterbatasan, dan memastikan analisis
  dilakukan dengan metodologi yang benar.
\item
  \textbf{Pengenalan Pola:} Kepekaan alami untuk melihat pola, tren,
  atau anomali dalam kumpulan data yang kompleks.
\end{itemize}

\section{2) H --- Heart (Minat Profesional \& Gairah
Intelektual)}\label{h-heart-minat-profesional-gairah-intelektual}

\begin{itemize}
\tightlist
\item
  \textbf{Data Science \& Machine Learning:} Ketertarikan mendalam pada
  proses mengekstrak makna dari data, baik melalui analisis statistik
  maupun pemodelan prediktif.
\item
  \textbf{Visualisasi \& Data Storytelling:} Gairah untuk mengubah angka
  mentah menjadi cerita visual yang mudah dipahami dan dapat
  ditindaklanjuti oleh non-teknisi.
\item
  \textbf{Pemecahan Masalah Pragmatis:} Antusiasme dalam menggunakan
  data untuk menjawab pertanyaan spesifik atau memecahkan tantangan
  bisnis/sistem yang nyata.
\item
  \textbf{Analisis dalam Konteks STI:} Minat khusus dalam menerapkan
  data science untuk mengoptimalkan sistem informasi, memahami proses
  bisnis, atau menganalisis perilaku pengguna (user behavior).
\end{itemize}

\section{3) A --- Abilities (Kemampuan
Andal)}\label{a-abilities-kemampuan-andal}

\begin{itemize}
\tightlist
\item
  \textbf{Analisis Data Kuantitatif:} Mampu melakukan data wrangling,
  analisis statistik deskriptif, dan pemodelan prediktif dasar.
\item
  \textbf{Pemrograman Data \& Database:} Terampil menggunakan Python dan
  SQL untuk mengambil, memanipulasi, dan menganalisis data.
\item
  \textbf{Visualisasi Data:} Mampu membuat dasbor dan visualisasi yang
  efektif menggunakan tools seperti Tableau, Looker Studio, atau
  Matplotlib/Seaborn.
\item
  \textbf{Analisis Sistem (Dasar):} Mampu memetakan alur proses (konteks
  STI) yang membantu memberikan konteks pada analisis data.
\item
  \textbf{Riset \& Pembelajaran Mandiri:} Sangat terampil dalam
  mempelajari library, algoritma, atau tools data science baru secara
  mandiri (misal: lewat Kaggle atau kursus online).
\end{itemize}

\section{4) P --- Personality (Gaya Kerja
Profesional)}\label{p-personality-gaya-kerja-profesional}

\begin{itemize}
\tightlist
\item
  \textbf{Metodis \& Terstruktur:} Bekerja dengan alur kerja yang jelas
  (CRISP-DM), mulai dari pemahaman masalah, penyiapan data, pemodelan,
  hingga evaluasi.
\item
  \textbf{Analitis \& Detail:} Sangat teliti dalam proses data cleaning
  dan validasi, sadar bahwa kesalahan kecil pada data bisa berakibat
  fatal pada kesimpulan.
\item
  \textbf{Mandiri \& Proaktif:} Tidak menunggu data ``bersih''
  diberikan; aktif mencari data, membersihkannya, dan mengambil
  inisiatif untuk melakukan eksplorasi.
\item
  \textbf{Reflektif:} Cenderung mengambil waktu untuk memikirkan ``apa
  arti'' dari hasil analisis, tidak hanya melaporkan angkanya.
\end{itemize}

\section{5) E --- Experiences (Pengalaman
Pembentuk)}\label{e-experiences-pengalaman-pembentuk}

\begin{itemize}
\tightlist
\item
  \textbf{Proyek Akademis STI:} Melatih kemampuan dasar analisis proses,
  basis data, dan arsitektur sistem (misal: mata kuliah Literasi Data
  dan AI).
\item
  \textbf{Pengalaman ``Benturan'':} Momen ketika mengerjakan proyek
  akademis STI dan menyadari bahwa passion terbesarnya adalah pada
  analisis data-nya, bukan pada perancangan sistemnya.
\item
  \textbf{Pembelajaran Mandiri Data Science:} Proses proaktif belajar
  Python, SQL, dan Machine Learning di luar kurikulum wajib STI (misal:
  lewat kursus daring, Kaggle), yang membangun disiplin dan portofolio.
\end{itemize}

\section{6) Piagam Diri (Self-Charter)}\label{piagam-diri-self-charter}

Misi Profesional:

Menjadi jembatan antara data teknis dan kebutuhan bisnis/sistem, dengan
menerjemahkan data mentah menjadi wawasan strategis menggunakan analisis
data yang logis, metodis, dan berintegritas.

Nilai Inti:

Logika, Integritas (Data), Akurasi, Keingintahuan, Pertumbuhan.

Peran Ideal:

Data Scientist, Data Analyst, Business Intelligence Analyst, Systems
Analyst (Data-focused).

Kompas Keputusan:

\begin{enumerate}
\def\labelenumi{(\arabic{enumi})}
\tightlist
\item
  Apakah ini didukung oleh data? (2) Apakah analisisnya akurat dan
  logis? (3) Apakah ini memecahkan masalah nyata bagi sistem/pengguna?
  (4) Apakah saya bisa belajar hal baru dari proses ini?

  Janji Profesional:

  Untuk selalu memulai dengan pertanyaan ``apa masalahnya'', memvalidasi
  data sebelum menganalisis, dan menyajikan wawasan dengan jujur dan
  jelas.
\end{enumerate}

\section{7) Narasi Diri (versi 90
detik)}\label{narasi-diri-versi-90-detik}

\begin{quote}
``Saya Faizal Ali, biasa dipanggil Faiz, seorang mahasiswa Sistem dan
Teknologi Informasi (STI). Di STI, saya belajar pentingnya merancang
sistem yang efisien. Namun, saya menemukan `benturan'---passion terbesar
saya ternyata ada pada data di dalam sistem itu.

Kekuatan saya ada pada pemikiran analitis dan metodis. Saya menikmati
proses mengurai data mentah, membersihkannya, dan menemukan pola
tersembunyi. Pengalaman saya di proyek akademis STI dan pembelajaran
mandiri data science mengajari saya satu hal: data adalah cara terbaik
untuk memahami sistem.

Kini, misi saya adalah menggabungkan kedua dunia itu: menggunakan data
science untuk menjawab `mengapa' sebuah sistem bekerja, dan `bagaimana'
kita bisa membuatnya lebih baik.''
\end{quote}

\section{8) Rencana Aksi 90 Hari
(SMART)}\label{rencana-aksi-90-hari-smart}

\emph{(Sebagai mahasiswa, aksi saya fokus pada pengembangan portofolio
data)}

\begin{enumerate}
\def\labelenumi{\arabic{enumi}.}
\tightlist
\item
  \textbf{Selesaikan 1 proyek portofolio end-to-end.}\\
  \emph{Outcome:} 1 repositori GitHub berisi analisis data (dari
  scraping/cleaning hingga visualisasi/model) pada dataset baru.\\
  \emph{Due:} T-90 hari.
\item
  \textbf{Pelajari 1 tools visualisasi data secara mendalam.}\\
  \emph{Outcome:} Menerbitkan 2 dasbor baru di Tableau Public untuk
  mengeksplorasi fitur-fitur yang lebih advanced.\\
  \emph{Due:} T-60 hari.
\item
  \textbf{Identifikasi 1 topik Tugas Akhir (TA) yang menggabungkan STI
  \& Data Science.}\\
  \emph{Outcome:} 1 proposal singkat (1 halaman) berisi rumusan masalah
  TA dan telah dikonsultasikan dengan 1 calon dosen pembimbing.\\
  \emph{Due:} T-75 hari.
\end{enumerate}

\bookmarksetup{startatroot}

\chapter{UTS-5 My Personal Reviews}\label{uts-5-my-personal-reviews}

Berikut cara saya melakukan review: mengguan chatGPT, saya mengattach
\href{skor_uts.pdf}{file promt ChatGPT}, disertai perintah :``self
assess uts-1 sanpai uts-5 dari URL
`https://ii-2100.github.io/all-about-me/'\,''

ChatGPT melakukan self-assessment UTS-1 s.d. UTS-5 langsung dari laman
yang Anda berikan dan menilai memakai rubrik tugas UTS (skala 1--5 per
kriteria). Rekap skor siap diunduh sebagai CSV:
\href{sandbox:/mnt/data/UTS_self_assessment.csv}{Download CSV
ringkasan}.

\bookmarksetup{startatroot}

\chapter{Hasil Self-Assessment UTS (URL:
ii-2100.github.io/all-about-me)}\label{hasil-self-assessment-uts-url-ii-2100.github.ioall-about-me}

\section{Identifikasi}\label{identifikasi}

\begin{itemize}
\tightlist
\item
  Nama \& NIM penulis: \textbf{Armein Z. R. Langi -- 131902360} (tertera
  di halaman depan portofolio).
  (\href{https://ii-2100.github.io/all-about-me/}{II 2100})
\item
  Penilai: \textbf{Self-assessment (Armein Z. R. Langi)}
\item
  Catatan cakupan: halaman beranda memuat ``About Me''; navigasi ke ``My
  Songs for You'', ``My Stories for You'', ``My Shapes'', dan ``My
  Personal Reviews'' tersedia.
  (\href{https://ii-2100.github.io/all-about-me/}{II 2100})
\end{itemize}

\section{Tinjauan Umum}\label{tinjauan-umum}

\begin{itemize}
\tightlist
\item
  \textbf{UTS-1 (All About Me)} hadir di beranda (``Selamat Berjumpa /
  About Me''). Isi memperkenalkan identitas dan latar personal secara
  padat. (\href{https://ii-2100.github.io/all-about-me/}{II 2100})
\item
  \textbf{UTS-2 (My Songs for You)} memuat judul karya dan tautan audio,
  namun lirik/isi tidak ditampilkan di halaman (file audio tidak bisa
  saya akses dari sini), sehingga penilaian konten terbatas pada
  kelengkapan presentasi.
  (\href{https://ii-2100.github.io/all-about-me/My_Song_for_You/index.html}{II
  2100})
\item
  \textbf{UTS-3 (My Stories for You)} berisi tautan ke beberapa cerita;
  konten naratif kuat, terutama pada kisah keluarga ``Gaun Pengantin
  Gladys'' serta kisah inspiratif lainnya.
  (\href{https://ii-2100.github.io/all-about-me/My_Stories_for_You/index.html}{II
  2100})
\item
  \textbf{UTS-4 (My SHAPE)} saat ini halaman kosong/tanpa isi, hanya
  navigasi.
  (\href{https://ii-2100.github.io/all-about-me/My_Shapes/index.html}{II
  2100})
\item
  \textbf{UTS-5 (My Personal Reviews)} berisi metode/tautan panduan
  review, namun belum ada contoh \textbf{review personal} yang lengkap
  terhadap sebuah pesan/teks.
  (\href{https://ii-2100.github.io/all-about-me/My_Personal_Reviews/index.html}{II
  2100})
\end{itemize}

\begin{center}\rule{0.5\linewidth}{0.5pt}\end{center}

\section{Tinjauan Spesifik + Skor
(1--5)}\label{tinjauan-spesifik-skor-15}

\subsection{UTS-1 --- All About Me (di
beranda)}\label{uts-1-all-about-me-di-beranda}

\textbf{Skor per kriteria:} Orisinalitas \textbf{3}, Keterlibatan
\textbf{3}, Humor \textbf{1}, Wawasan/Insight \textbf{3} → \textbf{Total
10/20 (50\%)}. \textbf{Alasan singkat:} Perkenalan diri faktual dan
jelas, namun gaya masih informatif ketimbang naratif (minim humor dan
moment ``hook''). (\href{https://ii-2100.github.io/all-about-me/}{II
2100}) \textbf{Saran perbaikan:} Buka dengan anekdot pribadi (1--2
paragraf) yang ``mengikat'' (mis. titik balik karier/keluarga),
tambahkan satu momen humor ringan, lalu tutup dengan refleksi singkat
tentang nilai/visi diri agar aspek insight naik.

\subsection{UTS-2 --- My Songs for You}\label{uts-2-my-songs-for-you-1}

\textbf{Skor per kriteria:} Orisinalitas \textbf{2}, Keterlibatan
\textbf{2}, Humor \textbf{1}, Inspirasi \textbf{2} → \textbf{Total 7/20
(35\%)}. \textbf{Alasan singkat:} Halaman menampilkan judul lagu ``River
in my Mind'', ``Heaven on Earth'', namun tanpa lirik/cerita di balik
lagu sehingga sulit menilai aspek pesan, humor, dan inspirasi.
(\href{https://ii-2100.github.io/all-about-me/My_Song_for_You/index.html}{II
2100}) \textbf{Saran perbaikan:} Tambahkan lirik lengkap, 1 paragraf
cerita proses kreatif, dan 2--3 kalimat ``pesan untukmu'' agar inspirasi
terbaca; sertakan player/tautan yang dapat diputar langsung + fallback
transkrip.

\subsection{UTS-3 --- My Stories for
You}\label{uts-3-my-stories-for-you-1}

\textbf{Skor per kriteria:} Orisinalitas \textbf{5}, Keterlibatan
\textbf{5}, Pengembangan Narasi \textbf{4}, Inspirasi \textbf{5} →
\textbf{Total 19/20 (95\%)}. \textbf{Alasan singkat:} Cerita ``Gaun
Pengantin Gladys'' dkk sangat personal, emosional, dan inspiratif; ritme
bertutur hidup serta detail situasional kuat (konten ramu unsur
penebusan/keluarga/iman).
(\href{https://ii-2100.github.io/all-about-me/My_Stories_for_You/index.html}{II
2100}) \textbf{Saran perbaikan:} Tambah ``lead'' 2--3 kalimat yang
merangkum pesan kunci tiap cerita; akhiri dengan ajakan/refleksi 1--2
kalimat agar resonansi ke pembaca makin jelas.

\subsection{UTS-4 --- My SHAPE}\label{uts-4-my-shape}

\textbf{Skor per kriteria:} Orisinalitas \textbf{1}, Keterlibatan
\textbf{1}, Pengembangan \textbf{1}, Inspirasi \textbf{1} →
\textbf{Total 4/20 (20\%)}. \textbf{Alasan singkat:} Konten belum
tersedia.
(\href{https://ii-2100.github.io/all-about-me/My_Shapes/index.html}{II
2100}) \textbf{Saran perbaikan (prioritas):}

\begin{enumerate}
\def\labelenumi{\arabic{enumi}.}
\tightlist
\item
  Tampilkan rekap SHAPE (Spiritual gifts, Heart, Abilities, Personality,
  Experiences) dalam 1 tabel ringkas + 3--5 butir bukti;
\item
  Narasikan 2 pengalaman kunci (±150 kata/each) yang menunjukkan
  ``signature strengths'';
\item
  Tutup dengan 3 rencana aksi konkret (who/what/when).
\end{enumerate}

\subsection{UTS-5 --- My Personal
Reviews}\label{uts-5-my-personal-reviews-1}

\textbf{Skor per kriteria:} Pemahaman Konsep \textbf{2}, Analisis Kritis
\textbf{1}, Argumentasi (Logos) \textbf{1}, Etos \& Empati \textbf{2},
Rekomendasi \textbf{1} → \textbf{Total 7/25 (28\%)}. \textbf{Alasan
singkat:} Halaman berisi metode cara menilai, tetapi \textbf{belum ada}
contoh \textbf{review personal} yang lengkap terhadap sebuah pesan/teks
sehingga aspek analisis-argumentasi tak terbaca.
(\href{https://ii-2100.github.io/all-about-me/My_Personal_Reviews/index.html}{II
2100}) \textbf{Saran perbaikan:} Pilih 1 karya personal (mis.
UTS-1/2/3), tulis review 400--600 kata: ringkas pesan, nilai dengan
rubrik, berikan 2--3 bukti kutipan, evaluasi etos/empati, lalu tutup
dengan rekomendasi perbaikan spesifik.

\begin{center}\rule{0.5\linewidth}{0.5pt}\end{center}

\section{Rekap Skor (ringkas)}\label{rekap-skor-ringkas}

\begin{itemize}
\tightlist
\item
  \textbf{UTS-1:} 10/20 → \textbf{50\%}
\item
  \textbf{UTS-2:} 7/20 → \textbf{35\%}
\item
  \textbf{UTS-3:} 19/20 → \textbf{95\%}
\item
  \textbf{UTS-4:} 4/20 → \textbf{20\%}
\item
  \textbf{UTS-5:} 7/25 → \textbf{28\%}
\end{itemize}

CSV lengkap sudah saya siapkan untuk dokumentasi dan olah lanjut:
\href{sandbox:/mnt/data/UTS_self_assessment.csv}{Download CSV
ringkasan}.

\section{Langkah Perbaikan Cepat (prioritas 1
minggu)}\label{langkah-perbaikan-cepat-prioritas-1-minggu}

\begin{enumerate}
\def\labelenumi{\arabic{enumi}.}
\tightlist
\item
  \textbf{Lengkapi UTS-4 (My SHAPE)} sesuai saran di atas;
\item
  \textbf{Tambahkan lirik + sinopsis} untuk tiap lagu di UTS-2 agar bisa
  dinilai isi/inspirasinya;
\item
  \textbf{Tulis 1 review personal penuh} (UTS-5) memakai rubrik (pilih
  salah satu karya Anda sendiri sebagai objek);
\item
  Revisi UTS-1 dengan pembuka anekdot + refleksi nilai diri.
\end{enumerate}

Jika Anda mau, saya bisa bantu merapikan UTS-4 (tabel SHAPE + narasi)
dan membuat kerangka cepat untuk \textbf{review} di UTS-5 dari salah
satu karya Anda.

\bookmarksetup{startatroot}

\chapter{UAS-1 My Concepts}\label{uas-1-my-concepts}

Mau hidup epik ? \href{lifestory.pdf}{Write your Life Story}

Apa itu berkonsep?

\url{https://youtu.be/QVfUlVBO80U?si=yM6q_rwV9rcDBbu7}

\bookmarksetup{startatroot}

\chapter{UAS-3 My Opinions}\label{uas-3-my-opinions}

SApa itu beropini? \href{BM\%20Opini.mp4}{Opini Berpengaruh}

Bagiamana menjaadi menarik? \href{./Interesting.mp4}{Menjadi Menarik}

\bookmarksetup{startatroot}

\chapter{UAS-3 My Innovations}\label{uas-3-my-innovations}

\bookmarksetup{startatroot}

\chapter{UAS-4 My Knowledge}\label{uas-4-my-knowledge}

Cara saya mengkomunikasikan sebuah pengatahuan sebagai petunjuk bagi
orang lain 1) saya tulis
\href{Rekomendasi\%20Presentasi\%20Efektif(Contoh\%20Makalah).pdf}{makalah
sebagai bahan utama} 2) lalu saya buat
\href{Contoh\%20Transkrip\%20Presentasi.pdf}{transkrip ucapan lisan} 3)
kemudian saya kembangkan
\href{Rekomendasi\%20Presentasi\%20(Contoh\%20Slides).pdf}{slide
pendukung trnsskrip} 4) lalu saya memproduksivideo audio visual
\url{https://youtu.be/ZbghfMvnPZc} \url{https://youtu.be/ZbghfMvnPZc}

\bookmarksetup{startatroot}

\chapter{UAS-5 My Professional
Reviews}\label{uas-5-my-professional-reviews}

\section{5.0.1 1 --- All About Me}\label{all-about-me}

\begin{longtable}[]{@{}
  >{\raggedright\arraybackslash}p{(\linewidth - 4\tabcolsep) * \real{0.1368}}
  >{\centering\arraybackslash}p{(\linewidth - 4\tabcolsep) * \real{0.0526}}
  >{\raggedright\arraybackslash}p{(\linewidth - 4\tabcolsep) * \real{0.8105}}@{}}
\toprule\noalign{}
\begin{minipage}[b]{\linewidth}\raggedright
Kriteria
\end{minipage} & \begin{minipage}[b]{\linewidth}\centering
Nilai
\end{minipage} & \begin{minipage}[b]{\linewidth}\raggedright
Catatan Singkat
\end{minipage} \\
\midrule\noalign{}
\endhead
\bottomrule\noalign{}
\endlastfoot
Orisinalitas & 5 & Ide dan narasi sangat personal, mencerminkan
identitas unik penulis. \\
Keterlibatan & 4 & Gaya tulisan menarik, namun bisa lebih diperkuat
untuk membangun koneksi emosional yang lebih dalam. \\
Humor & 3 & Fokus pada nada reflektif dan serius; humor bukan elemen
utama dalam tulisan ini. \\
Wawasan & 5 & Menunjukkan kesadaran diri yang tinggi dan koneksi yang
jelas antara nilai pribadi dengan tujuan. \\
\end{longtable}

Rata-rata : 4.25 (A) Komentar: Fondasi yang sangat kuat dengan narasi
personal yang otentik. Refleksi diri yang ditampilkan menunjukkan
pemikiran yang matang.

\section{5.0.2 2 --- My Song for You}\label{my-song-for-you}

\begin{longtable}[]{@{}
  >{\raggedright\arraybackslash}p{(\linewidth - 4\tabcolsep) * \real{0.1368}}
  >{\centering\arraybackslash}p{(\linewidth - 4\tabcolsep) * \real{0.0526}}
  >{\raggedright\arraybackslash}p{(\linewidth - 4\tabcolsep) * \real{0.8105}}@{}}
\toprule\noalign{}
\begin{minipage}[b]{\linewidth}\raggedright
Kriteria
\end{minipage} & \begin{minipage}[b]{\linewidth}\centering
Nilai
\end{minipage} & \begin{minipage}[b]{\linewidth}\raggedright
Catatan Singkat
\end{minipage} \\
\midrule\noalign{}
\endhead
\bottomrule\noalign{}
\endlastfoot
Orisinalitas & 4 & Pilihan lagu unik dan interpretasi personal, namun
narasi bisa lebih diperkaya dengan konteks spesifik. \\
Keterlibatan & 4 & Mampu membawa pembaca ke dalam suasana reflektif
lagu-lagu yang dipilih. \\
Humor & 3 & Nada tulisan cenderung serius dan introspektif, sesuai
dengan tema lagu. \\
Inspirasi & 5 & Pesan penutup tentang ``menjadi pusat, bukan korban''
sangat kuat dan inspiratif. \\
\end{longtable}

Rata-rata : 4.00 (A) Komentar: Berhasil menunjukkan kedalaman emosional
dan kemampuan untuk menarik makna personal dari sebuah karya seni.

\section{5.0.3 3 --- My Stories for You}\label{my-stories-for-you}

\begin{longtable}[]{@{}
  >{\raggedright\arraybackslash}p{(\linewidth - 4\tabcolsep) * \real{0.1810}}
  >{\centering\arraybackslash}p{(\linewidth - 4\tabcolsep) * \real{0.0476}}
  >{\raggedright\arraybackslash}p{(\linewidth - 4\tabcolsep) * \real{0.7714}}@{}}
\toprule\noalign{}
\begin{minipage}[b]{\linewidth}\raggedright
Kriteria
\end{minipage} & \begin{minipage}[b]{\linewidth}\centering
Nilai
\end{minipage} & \begin{minipage}[b]{\linewidth}\raggedright
Catatan Singkat
\end{minipage} \\
\midrule\noalign{}
\endhead
\bottomrule\noalign{}
\endlastfoot
Orisinalitas & 5 & Gaya penceritaan sangat otentik dan unik, menciptakan
pengalaman membaca yang sangat personal. \\
Keterlibatan & 5 & Narasi ``show, don't tell'' dieksekusi sempurna,
sangat berhasil menarik pembaca ke dalam cerita. \\
Pengembangan Narasi & 5 & Struktur cerita dan refleksi sangat solid;
alur dan tempo penceritaan terkontrol dengan sangat baik. \\
Inspirasi & 5 & Pesan tentang agensi diri dan tanggung jawab personal
tersampaikan secara mendalam dan berkesan. \\
\end{longtable}

Rata-rata : 5.00 (A+) Komentar: Sebuah karya tulis yang luar biasa.
Matang secara naratif, emosional, dan filosofis. Titik puncak dari seri
ini.

\section{5.0.4 4 --- My SHAPE}\label{my-shape}

\begin{longtable}[]{@{}
  >{\raggedright\arraybackslash}p{(\linewidth - 4\tabcolsep) * \real{0.1373}}
  >{\centering\arraybackslash}p{(\linewidth - 4\tabcolsep) * \real{0.0490}}
  >{\raggedright\arraybackslash}p{(\linewidth - 4\tabcolsep) * \real{0.8137}}@{}}
\toprule\noalign{}
\begin{minipage}[b]{\linewidth}\raggedright
Kriteria
\end{minipage} & \begin{minipage}[b]{\linewidth}\centering
Nilai
\end{minipage} & \begin{minipage}[b]{\linewidth}\raggedright
Catatan Singkat
\end{minipage} \\
\midrule\noalign{}
\endhead
\bottomrule\noalign{}
\endlastfoot
Orisinalitas & 5 & Analisis terasa lahir dari refleksi nyata dan
pengalaman pribadi, bukan sekadar mengisi template. \\
Keterlibatan & 5 & Struktur yang logis dan benang merah yang kuat
membuat dokumen ini sangat menarik dan mudah diikuti. \\
Keautentikan & 4 & Sangat otentik, namun bisa diperkuat lagi dengan
contoh/bukti konkret dari pengalaman yang disebutkan. \\
Inspirasi & 5 & Bagian ``Piagam Diri'' dan ``Kompas Keputusan'' sangat
inspiratif dan dapat ditindaklanjuti (actionable). \\
\end{longtable}

Rata-rata : 4.75 (A) Komentar: Analisis diri yang sangat komprehensif
dan strategis. Berhasil mengubah refleksi menjadi sebuah panduan kerja
yang jelas.

\section{5.0.5 Rangkuman Nilai Akhir}\label{rangkuman-nilai-akhir}

\begin{longtable}[]{@{}lcc@{}}
\toprule\noalign{}
UTS & Rata-rata & Tingkat \\
\midrule\noalign{}
\endhead
\bottomrule\noalign{}
\endlastfoot
UTS-1 All About Me & 4.25 & A \\
UTS-2 My Song for You & 4.00 & A \\
UTS-3 My Stories for You & 5.00 & A+ \\
UTS-4 My SHAPE & 4.75 & A \\
\textbf{Rata-rata Keseluruhan} & \textbf{4.50 / 5.00 (A)} & \\
\end{longtable}

\section{5.0.6 Kesimpulan}\label{kesimpulan}

Seluruh karya pada repositori \textbf{All About Me} menunjukkan
\textbf{perkembangan kepribadian dan refleksi diri yang konsisten,
cerdas, dan mendalam.} Setiap UTS memperlihatkan kematangan berpikir
serta keaslian ekspresi yang khas. Tulisan-tulisan Anda tidak hanya
memenuhi rubrik akademik, tetapi juga berhasil menjadi potret perjalanan
intelektual dan personal yang otentik.

\begin{quote}
``Menjadi reflektif bukan berarti berhenti, melainkan melangkah dengan
lebih sadar akan arah dan makna.''
\end{quote}

\bookmarksetup{startatroot}

\chapter{Summary}\label{summary}

In summary, this book has no content whatsoever.

\bookmarksetup{startatroot}

\chapter*{References}\label{references}
\addcontentsline{toc}{chapter}{References}

\markboth{References}{References}

\phantomsection\label{refs}




\end{document}
